\ifnum\strcmp{\jobname}{HLM}=0
	\input{../set/preclass}
	\documentclass{article}
	\input{../set/default}
	\input{../set/user}
	\makeatletter
%-------------------------------------------------------------------------------------------------------
\usepackage{enumitem}
\setlist[enumerate, 1]
{	fullwidth,
	label = \arabic*., 
	font = \textup, 
	itemindent=2em
}
\setlist[enumerate, 2]
{	fullwidth,
	label = \arabic*., 
	font = \textup, 
	itemindent=2em
}
%-------------------------------------------------------------------------------------------------------
\pagestyle{fancy}
\renewcommand{\headrulewidth}{0pt}
\lhead{}
\chead{}
\rhead{}
\lfoot{\small{第\chinese{section}节}}
\cfoot{\small{第\thepage 页~共\pageref{LastPage}页}}
\rfoot{}
%-------------------------------------------------------------------------------------------------------
\title{\textbf{基于统计学方法的企业经营优化}}
\author{}
\date{}
%-------------------------------------------------------------------------------------------------------
\makeatother

	\begin{document}
\fi
%-------------------------------------------------------------------------------------------------------
\clearpage
\section{多层线性模型}
\subsection{模型的建立与求解}
根据收集到的数据将顾客按年龄分成5组\(t_1..t_2\)。各公司年龄结构数据如下。
\begin{table}[htbp]
	\caption{各公司顾客年龄结构}
	\begin{longtabu}to\linewidth{@{}X[c]|*6{X[c]}@{}}
		\toprule
		\diagbox{年龄段}{比例}{公司}&三星      & 苹果        & 中兴       & 华为         & 小米        & 诺基亚\\\midrule
		18以下  & 5  & 3  & 3  & 14 & 12  & 3  \\
		19-28 & 20 & 27 & 24 & 26 & 32  & 27 \\
		29-35 & 40 & 40 & 36 & 28 & 28  & 38 \\
		36-45 & 34 & 26 & 32 & 19 & 20  & 20 \\
		45以上  & 1  & 4  & 5  & 13 & 8   & 12 \\ \bottomrule
	\end{longtabu}
\end{table}
\par\indent 之前的模型只是简单的对各环节投入与利润进行回归分析,考虑到不同年龄段顾客消费水平不同,有必要建立多层线性模型,将模型分为2层,先考察各环节投入对顾客年龄结构的影响,再考察不同年龄结构对利润的影响。
\subsubsection{第1层线性回归}
年龄结构与各环节投入的关系如下。
\begin{align}
	\begin{bmatrix}
		\hat{t_1}\\
		\hat{t_2}\\
		\hat{t_3}\\
		\hat{t_4}\\
		\hat{t_5}
	\end{bmatrix}
	=
	\begin{bmatrix}
		-0.032913858 & 0.003054487  & 0.085744675  & 0.013313007  & -0.134757241 \\
		-0.030339745 & 0.002899403  & 0.067673153  & 0.013130783  & -0.110888763 \\
		0.025872388  & -0.003803748 & -0.063629188 & -0.000516196 & 0.106016012  \\
		0.038698689  & -0.00593397  & -0.094196714 & -0.002557721 & 0.160426717  \\
		-0.001317475 & 0.003783827  & 0.004408075  & -0.023369872 & -0.020796726
	\end{bmatrix}
	\begin{bmatrix}
		x_1\\
		x_2\\
		x_3\\
		x_4\\
		x_5
	\end{bmatrix}
	+
	\begin{bmatrix}
		9.642428283 \\
		29.9399034  \\
		30.85210828 \\
		24.40001684 \\
		5.1655432 
\end{bmatrix}
\end{align}
\par\indent 残差矩阵如下。
\begin{align}
	\bm{R}=
	\begin{bmatrix}
		1.0004&0.2819&-0.0048&-0.0052&0&0  \\
		0.0399&-0.2267&0.0517&-0.0127&0&0  \\
		-0.6644&0.2190&0.0229&0.0190&0&0 \\
		0.3439&-0.3055&-0.0304&0.0164&0&0  \\
		-0.7199&0.0312&-0.0395&-0.0176&0&0
	\end{bmatrix}
\end{align}
\subsubsection{第2层线性回归}
年龄结构与利润关系如下。
\begin{align}
	\hat{y}=-81.88712132+0.281643264t_1+0.995823947t_2-1.390363326t_3-2.346812302t_4+0.427210113t_5
\end{align}
\par\indent 各项残差如下。
\begin{align}
	\begin{bmatrix}
		r_0 & r_1 & r_2 & r_3 & r_4 & r_5
	\end{bmatrix}
	=
	\begin{bmatrix}
		-2.91E-11 &	-2.18E-11 &	-1.14E-12 &	-9.09E-13 & 2.27E-12 & 3.64E-12
	\end{bmatrix}
\end{align}
\par\indent 可以看到年龄段在19到28的顾客人群消费水平较高,导致\(t_2\)回归系数较大。
\\\indent 理论上应该改变各环节投入使得消费水平最高的顾客人群占比最高,但实际上还有市场竞争等多方面因素影响,而诺基亚如果要选择青年和老年顾客群体较大,应调整其各环节投入如下。
\begin{align}
	\begin{bmatrix}
		x_1\\
		x_2\\
		x_3\\
		x_4\\
		x_5
	\end{bmatrix}
	=
	\begin{bmatrix}
		13.78\\
		55.4\\
		11.1\\
		3.7\\
		4.5
	\end{bmatrix}
\end{align}
\par\indent 根据第2层线性模型理论上利润为30454百万美元。
\subsection{模型的检验与分析}
\subsubsection{相关性分析}
2层的拟合优度分别为0.874和0.891。相关程度较高。
\subsubsection{F检验}
同理,第1层的\(F=95.16\)。第二层的\(F=33.38\)。均能通过显著性检验。
%-------------------------------------------------------------------------------------------------------
\ifnum\strcmp{\jobname}{HLM}=0
	\end{document}
\fi