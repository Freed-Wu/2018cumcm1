\ifnum\strcmp{\jobname}{evaluate}=0
	\input{../set/preclass}
	\documentclass{article}
	\input{../set/default}
	\input{../set/user}
	\makeatletter
%-------------------------------------------------------------------------------------------------------
\usepackage{enumitem}
\setlist[enumerate, 1]
{	fullwidth,
	label = \arabic*., 
	font = \textup, 
	itemindent=2em
}
\setlist[enumerate, 2]
{	fullwidth,
	label = \arabic*., 
	font = \textup, 
	itemindent=2em
}
%-------------------------------------------------------------------------------------------------------
\pagestyle{fancy}
\renewcommand{\headrulewidth}{0pt}
\lhead{}
\chead{}
\rhead{}
\lfoot{\small{第\chinese{section}节}}
\cfoot{\small{第\thepage 页~共\pageref{LastPage}页}}
\rfoot{}
%-------------------------------------------------------------------------------------------------------
\title{\textbf{基于统计学方法的企业经营优化}}
\author{}
\date{}
%-------------------------------------------------------------------------------------------------------
\makeatother

	\begin{document}
\fi
%-------------------------------------------------------------------------------------------------------
\clearpage
\section{模型的评价与改进}
\subsection{优点}
\begin{enumerate}
	\item 针对不同环节设立了不同的指标,系统反应了利润率的各种影响因素,考虑周到全面,采用时间序列回归分析考虑到了研发设计的投入对于利润率的贡献具有滞后性,更加符合实际情况;
	\item 针对各公司企业定位和预期市场可以合理调整各环节投入产出比,进而推出不同的投入优化模型,具有良好的针对性和扩展性;
\end{enumerate}
\subsection{缺点}
\begin{enumerate}
	\item 实际上各个环节对于利润的贡献都具有一定滞后性,由于历史数据和计算机计算能力的限制仅考虑了具有较大时间常数的研发设计环节的滞后性;
	\item 实际上公司投入分配会留有裕量来保证产品具有一定的灵活性,后期可以针对备用投资裕量完善模型,使之更符合实际情况;
	\item 考虑到实际公司经营时具有一定不可预知性,如其他公司的竞争、个别安全问题等,因此模型只是理论上的最优解,与实际情况有一定出入。
\end{enumerate}
\subsection{改进}
\begin{enumerate}
	\item 实际情况中各个环节的投入都应该具有一定滞后性,可以考虑时间序列得出各环节投入产出的滞后时间常数使模型更加贴近实际情况。
	\item 考虑到实际经营中一些不可抗力如个别安全时间等,可以建立安全环节的投入与安全事故概率的关系以及安全事故的影响模型,更加符合实际情况。
	\item 考虑到利润的复杂影响因素,可以优化模型算法比如利用神经网络算法,得出更优化更符合实际情况的模型解。
\end{enumerate}
\subsection{推广}
\begin{enumerate}
	\item 企业资金管理是企业经营与运营决策的基础,本质是有限资源的运用与管理。根据不同运用方,可以扩展到政府财政预算投入或者工程人员分配。通过分析各种事务包括一般公共服务、发展与改革事物、物价管理、人口与计划生育事务等对社会稳定和经济增长的不同贡献,合理推广本文模型调整财政预算支出分配,针对国家稳定性与经济发展的预期目标,可以获得国家财政税收分配的最优解。
\end{enumerate}
%-------------------------------------------------------------------------------------------------------
\ifnum\strcmp{\jobname}{evaluate}=0
	\end{document}
\fi
