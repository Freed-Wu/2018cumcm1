\ifnum\strcmp{\jobname}{PCA}=0
	\RequirePackage[l2tabu, orthodox]{nag}
\RequirePackage{ifxetex}
\RequireXeTeX

	\documentclass{article}
	\usepackage{media9}
\usepackage{pdfpages}
\usepackage{graphicx}
\graphicspath{{./src/}}
\usepackage{wallpaper}
\usepackage{wrapfig}
\usepackage{standalone}
\usepackage{tikz}
\usetikzlibrary{patterns}
\usetikzlibrary{arrows}
\usetikzlibrary{shapes}
\usetikzlibrary{chains}
\usetikzlibrary{mindmap}
\usetikzlibrary{graphs}
\usepackage{scsnowman}
\usepackage{tikzpeople}
\usepackage{tikzducks}
\usepackage{pifont}
\usepackage{marvosym}
%\usepackage{bbding}
%\usepackage{SIunits}
\usepackage{stmaryrd}
%-------------------------------------------------------------------------------------------------------
\usepackage{tabu}
\usepackage{longtable}
\usepackage{booktabs}
\usepackage{diagbox}
\usepackage{multicol}
\usepackage{multirow}
\usepackage{makecell}
\usepackage{fancybox}
\usepackage{colortbl}
\usepackage{tcolorbox}
\tcbuselibrary{skins}
\tcbuselibrary{breakable}
\tcbuselibrary{theorems}
%-------------------------------------------------------------------------------------------------------
\usepackage{amsmath}
\usepackage{amsthm}
\usepackage{amsfonts}
\usepackage{amssymb}
\usepackage{amsbsy}
\usepackage{amsopn}
\usepackage{amstext}
\usepackage{mathrsfs}
\usepackage{bm}
\usepackage{textcomp}
\usepackage{latexsym}
\usepackage{mathpazo}
\usepackage{exscale}
\usepackage{relsize}
%\usepackage{xymtex}
%-------------------------------------------------------------------------------------------------------
\usepackage{xcolor}
%-------------------------------------------------------------------------------------------------------
\usepackage[russian]{babel}
\let\captionsrussian\empty
\let\daterussian\empty
%-------------------------------------------------------------------------------------------------------
\usepackage{printlen}
\uselengthunit{mm}
%-------------------------------------------------------------------------------------------------------
\usepackage{fancyhdr}
\usepackage{geometry}
\usepackage{lastpage}
\usepackage{indentfirst}
\usepackage{setspace}
%-------------------------------------------------------------------------------------------------------
\usepackage{makeidx}
\makeindex
\usepackage{epigraph}
\usepackage{ean13isbn}
\usepackage{listings}
%-------------------------------------------------------------------------------------------------------
\usepackage{pagesel}

	\makeatletter
%-------------------------------------------------------------------------------------------------------
\xdefinecolor{dkgreen}{rgb}{0, 0.6, 0}
\xdefinecolor{gray}{rgb}{0.5, 0.5, 0.5}
\xdefinecolor{mauve}{rgb}{0.58, 0, 0.82}
\xdefinecolor{DeepSkyBlue}{RGB}{0, 104, 139}
\xdefinecolor{dkpurple}{rgb}{0.455, 0.204, 0.506}
\newif\ifTJ@green
\newif\ifTJ@orange
\newif\ifTJ@violet
\TJ@greentrue
\TJ@orangefalse
\TJ@violetfalse
\ifTJ@green
	\xdefinecolor{cvgreen}{HTML}{92D14F}
	\xdefinecolor{cvgray}{HTML}{D8E4BE}
	\xdefinecolor{cvtext}{HTML}{92909B}
\else
\fi
\ifTJ@orange
	\xdefinecolor{cvgreen}{RGB}{230, 140, 20}
	\xdefinecolor{cvgray}{RGB}{248, 216, 174}
	\xdefinecolor{cvtext}{RGB}{100, 100, 100}
\else
\fi
\ifTJ@violet
	\xdefinecolor{cvgreen}{RGB}{142, 10, 178}
	\xdefinecolor{cvgray}{RGB}{209, 181, 217}
	\xdefinecolor{cvtext}{RGB}{100, 100, 100}
\else
\fi
%-------------------------------------------------------------------------------------------------------
\usepackage{ctex}
\xeCJKsetup
{	CJKecglue = {\hskip 0.10em plus 0.05em minus 0.05em}, 
	CheckSingle = true
}
\setCJKmainfont
[	BoldFont = {Noto Serif CJK SC Bold}, 
	ItalicFont = {AR PL UKai HK}
]{Noto Serif CJK SC ExtraLight}
\setCJKsansfont
[	BoldFont = {Noto Sans S Chinese Medium}, 
%	ItalicFont = {Noto Sans S Chinese DemiLight}
]{SimSun}
\setCJKmonofont
[	BoldFont = {Adobe Heiti Std}, 
	ItalicFont = {Adobe Kaiti Std}
]{WenQuanYi Micro Hei Mono Light}
\setCJKfamilyfont{zhkesong}{MF KeSong (Noncommercial)}
\setCJKfamilyfont{zhhjsd}{MF TheGoldenEra (Noncommercial)}
\setCJKfamilyfont{zhlangsong}{MF LangSong (Noncommercial)}
\newcommand{\hjsd}{\CJKfamily{zhhjsd}}
\newcommand{\KeSong}{\CJKfamily{zhkesong}}
\newcommand{\langsong}{\CJKfamily{zhlangsong}}
\newfontfamily\KeSongf[Mapping = tex-text]{MF KeSong (Noncommercial)}
\newfontfamily\hjsdf[Mapping = tex-text]{MF TheGoldenEra (Noncommercial)}
\newfontfamily\langsongf[Mapping = tex-text]{MF LangSong (Noncommercial)}
\newfontfamily\SourceCodePro[Mapping = tex-text]{Source Code Pro}
\newfontfamily\monaco{Monaco}
%-------------------------------------------------------------------------------------------------------
\usepackage{microtype}
%-------------------------------------------------------------------------------------------------------
\usepackage
[	colorlinks = true, 
	linkcolor = gray, 
	citecolor = gray,
	backref=page
]{hyperref}
%-------------------------------------------------------------------------------------------------------
\renewcommand*\lstlistingname{代码}
\lstset
{	language = Matlab, 			%thelanguageofthecode
	basicstyle =\tiny\monaco, 		%thesizeofthefontsthatareusedforthecode
	numbers = left, 				%wheretoputtheline-numbers
	numberstyle = \tiny\color{gray}\monaco, 	%thestylethatisusedfortheline-numbers
	stepnumber = 1, 			%thestepbetweentwoline-numbers.Ifit's1, eachlinewillbenumbered
	numbersep = 5pt, 			%howfartheline-numbersarefromthecode
	backgroundcolor = \color{white}, %choosethebackgroundcolor.Youmustadd\usepackage{color}
	showspaces = false, 			%showspacesaddingparticularunderscores
	showstringspaces = false, 		%underlinespaceswithinstrings
	showtabs = false, 			%showtabswithinstringsaddingparticularunderscores
	frame = single, 				%addsaframearoundthecode
	rulecolor = \color{black}, 		%ifnotset, theframe-colormaybechangedonline-breakswithinnot-blacktext(e.g.commens(greenhere))
	tabsize = 4, 				%setsdefaulttabsizeto2spaces
	captionpos = b, 				%setsthecaption-positiontobottom
	breaklines = true, 			%setsautomaticlinebreaking
	breakatwhitespace = false, 	%setsifautomaticbreaksshouldonlyhappenatwhitespace
	title = \lstname, 				%showthefilenameoffilesincludedwith\lstinputlisting; alsotrycaptioninsteadoftitle
	keywordstyle = \color{blue}, 	%keywordstyle
	commentstyle = \color{dkgreen}, %commentstyle
	stringstyle = \color{mauve}, 	%stringliteralstyle
	escapeinside = ``, 			%ifyouwanttoaddLaTeXwithinyourcode
	morekeywords = {*, ...},		%ifyouwanttoaddmorekeywordstotheset
	columns=flexible
}
%-------------------------------------------------------------------------------------------------------
\tcbset
{	colframe = yellow! 50! green, 
	colback = white, 
	fonttitle = \bfseries, 
	left = 0mm, 
	right = 0mm, 
	top = 0mm, 
	bottom = 0mm, 
	boxsep = 0mm, 
 	toptitle = 0.5mm, 
 	bottomtitle = 0.5mm, 
	nobeforeafter, center title
}
\newtcbtheorem{strength}{\it 优点}
{	enhanced, breakable, 
	colback = white, 
	colframe = cyan, 
	colbacktitle = cyan, 
	attach boxed title to top left = 
	{	yshift = -2mm, 
		xshift = 5mm
	}, 
	boxed title style = {sharp corners}, 
	fonttitle = \sffamily\bfseries, 
	separator sign = {\it :}
}{}
\newtcbtheorem{weakness}{\it 缺点}
{	enhanced, breakable, 
	colback = white, 
	colframe = gray, 
	colbacktitle = gray, 
	attach boxed title to top left = 
	{	yshift = -2mm, 
		xshift = 5mm
	}, 
	boxed title style = {sharp corners}, 
	fonttitle = \sffamily\bfseries, 
	separator sign = {\it :}
}{}
%-------------------------------------------------------------------------------------------------------
\tikzstyle{level 2 concept} += [sibling angle = 40]
\tikzstyle{startstop} = 
[	rectangle, rounded corners, 
	minimum width = 2cm, 
	minimum height = 1cm, 
	text centered, 
	draw = black, 
	fill = red! 40
]
\tikzstyle{io} = 
[	trapezium, 
	trapezium left angle = 70, 
	trapezium right angle = 110, 
	minimum width = 2cm, 
	minimum height = 1cm, 
	text centered, 
	draw = black, 
	fill = blue! 40
]
\tikzstyle{process} = 
[	rectangle, 
	minimum width = 2cm, 
	minimum height = 1cm, 
	text centered, 
	draw = black, 
	fill = yellow! 50
]
\tikzstyle{decision} = 
[	diamond, 
	aspect = 3, 
	text centered, 
	draw = black, 
	fill = green! 30
]
\tikzstyle{arrow} = 
[	->, 
	> = stealth
]
%-------------------------------------------------------------------------------------------------------
\newcommand{\drawbarcode}[1]{\EANisbn[SC5a, ISBN = #1]}
%-------------------------------------------------------------------------------------------------------
\renewcommand{\thefootnote}{\fnsymbol{footnote}}
%-------------------------------------------------------------------------------------------------------
\usepackage[resetfonts]{cmap}
\usepackage{fancyvrb}
\begin{VerbatimOut}{main.cmap}
	%!PS-Adobe-3.0 Resource-CMap
	%%DocumentNeededResources: ProcSet (CIDInit)
	%%IncludeResource: ProcSet (CIDInit)
	%%BeginResource: CMap (TeX-OT1-0)
	%%Title: (TeX-OT1-0 TeX OT1 0)
	%%Version: 1.000
	%%EndComments
	/CIDInit /ProcSet findresource begin
	12 dict begin
	begincmap
	/CIDSystemInfo
	<< /Registry (TeX)
	/Ordering (OT1)
	/Supplement 0
	>> def
	/CMapName /TeX-OT1-0 def
	/CMapType 2 def
	1 begincodespacerange
	<00> <7F>
	endcodespacerange
	8 beginbfrange
	<00> <01> <0000>
	<09> <0A> <0000>
	<23> <26> <0000>
	<28> <3B> <0000>
	<3F> <5B> <0000>
	<5D> <5E> <0000>
	<61> <7A> <0000>
	<7B> <7C> <0000>
	endbfrange
	40 beginbfchar
	<02> <0000>
	<03> <0000>
	<04> <0000>
	<05> <0000>
	<06> <0000>
	<07> <0000>
	<08> <0000>
	<0B> <0000>
	<0C> <0000>
	<0D> <0000>
	<0E> <0000>
	<0F> <0000>
	<10> <0000>
	<11> <0000>
	<12> <0000>
	<13> <0000>
	<14> <0000>
	<15> <0000>
	<16> <0000>
	<17> <0000>
	<18> <0000>
	<19> <0000>
	<1A> <0000>
	<1B> <0000>
	<1C> <0000>
	<1D> <0000>
	<1E> <0000>
	<1F> <0000>
	<21> <0000>
	<22> <0000>
	<27> <0000>
	<3C> <0000>
	<3D> <0000>
	<3E> <0000>
	<5C> <0000>
	<5F> <0000>
	<60> <0000>
	<7D> <0000>
	<7E> <0000>
	<7F> <0000>
	endbfchar
	endcmap
	CMapName currentdict /CMap defineresource pop
	end
	end
	%%EndResource
	%%EOF
\end{VerbatimOut}
%-------------------------------------------------------------------------------------------------------
\makeatother

	\makeatletter
%-------------------------------------------------------------------------------------------------------
\usepackage{enumitem}
\setlist[enumerate, 1]
{	fullwidth,
	label = \arabic*., 
	font = \textup, 
	itemindent=2em
}
\setlist[enumerate, 2]
{	fullwidth,
	label = \arabic*., 
	font = \textup, 
	itemindent=2em
}
%-------------------------------------------------------------------------------------------------------
\pagestyle{fancy}
\renewcommand{\headrulewidth}{0pt}
\lhead{}
\chead{}
\rhead{}
\lfoot{\small{第\chinese{section}节}}
\cfoot{\small{第\thepage 页~共\pageref{LastPage}页}}
\rfoot{}
%-------------------------------------------------------------------------------------------------------
\title{\textbf{基于统计学方法的企业经营优化}}
\author{}
\date{}
%-------------------------------------------------------------------------------------------------------
\makeatother

	\begin{document}
\fi
%-------------------------------------------------------------------------------------------------------
\clearpage
\section{主成分分析模型}
\subsection{模型的建立与求解}
为了研究手机行业利润与各环节投入的关系,我们挑选了三星、苹果、中兴、华为、小米这5家该行业具有代表性的企业进行分析,考虑到第2题需要对诺基亚进行分析,再加上诺基亚的数据。因为数据不够全,只考虑了研发、生产、销售、管理、售后这5个环节。
\\\indent 定义变量\(y\)是每个公司的利润,定义变量\(x_1..x_5\)是5个环节分别的综合投入指标,定义变量\(x_{i1},x_{i2}\)是影响第\(i\)个环节的标志,分别是企业在各个环节财力与人力的投入。\(x_{ij},y\)已知,\(x_i\)未知。
\\\indent 分成2个环节进行分析:
\begin{enumerate}
\item 先建立综合评价模型,根据\(x_{i1},x_{i2}\)评价\(x_i\);
\item 对\(x_i,y\)进行回归分析以确定\(x_i\)和\(y\)的关系。
\end{enumerate}
\subsubsection{综合评价}
可以通过熵权法、层次分析法或神经网络确定综合标志从\(x_{i1},x_{i2}\)加权求和得到\(x_i\)的权值。
\begin{table}[htbp]
	\caption{各公司各环节投入及熵权}
	\begin{longtabu}to\linewidth{@{}X[1.5,c]|*6{X[c]}|X[0.5,c]@{}}
		\toprule
		\diagbox{标志}{数值}{公司} & 三星          & 苹果        & 中兴       & 华为         & 小米        & 诺基亚       & 熵权    \\ \midrule
		\makecell*{研发费用\\/百万美元}             & 13227.572   & 7636.795  & 2072.07  & 13555.4406 & 413.38    & 5060.858  & 0.823 \\
		\makecell*{研发人员\\/千人}              & 53.3        & 49        & 37       & 45         & 38        & 32        & 0.177 \\
		\makecell*{生产费用\\/百万美元}             & 103697.2026 & 102690.52 & 9919.624 & 55188.705  & 12153.372 & 33336.754 & 0.855 \\
		\makecell*{生产人员\\/千人}               & 20          & 23        & 24       & 25         & 25        & 28        & 0.116 \\
		\makecell*{销售费用\\/百万美元}             & 21233.734   & 21610.505 & 1184.04  & 8939.658   & 789.18    & 1282.963  & 0.891 \\
		\makecell*{销售人员\\/千人}               & 13          & 10        & 16       & 14         & 19        & 20        & 0.109 \\
		\makecell*{管理费用\\/百万美元}             & 16186.371   & 18523.29  & 871.585  & 4652.271   & 183.3904  & 4376.273  & 0.984 \\
		\makecell*{管理人员\\/千人}               & 6           & 9         & 7        & 6          & 9         & 10        & 0.016 \\
		\makecell*{售后费用\\/百万美元}             & 7658.068    & 9099.16   & 1019.59  & 3922.503   & 481.024   & 811.36    & 0.975 \\
		\makecell*{售后人员\\/千人}               & 7.7         & 9         & 16       & 10         & 9         & 10        & 0.025 \\ \bottomrule
	\end{longtabu}
\end{table}
\newpage
\par\indent 经过初步计算发现熵权法求出的权值与实际不符,经过推测原因可能是对于每个公司同一标志相差不明显导致熵权不具有代表性的缘故。又考虑到层次分析法过于主观,于是采用神经网络进行投入的综合评价。
\begin{table}[htbp]
	\caption{各公司各环节综合投入及利润}
	\begin{tabu}{@{}X[1.5,c]|*5X[c]|X[c]@{}}
	\toprule
	\diagbox{公司}{投入}{环节}	& 研发    & 生产  & 销售   & 管理   & 售后  &\makecell{利润/\\百万美元} \\\midrule
	三星  & 7.6   & 59.58 & 12.2 & 9.3  & 4.4 & 19594   \\
	苹果  & 4.7   & 63.2  & 13.3 & 11.4 & 5.6 & 53394   \\
	中兴  & 12.6  & 60.32 & 7.2  & 5.3  & 6.2 & 688     \\
	华为  & 14.86 & 60.5  & 9.8  & 5.1  & 4.3 & 7172    \\
	小米  & 2.75  & 80.85 & 5.25 & 1.22 & 3.2 & 810     \\
	诺基亚 & 9.98  & 65.74 & 2.53 & 8.63 & 1.6 & 26100  \\\bottomrule
	\end{tabu}
\end{table}
\par\indent 接下来为下文的主成分分析方便起见,将\(x_1..x_5\)标准化得到\(\tilde{x}_1..\tilde{x}_5\)。
\subsubsection{主成分分析}
因为有的环节是相互关联的,企业重视某一环节可能也会连带着对与其密切相关的其他环节增加投入,所以变量\(x_1..x_5\)之间可能并非独立的,为此,可以使用主成分分析或因子分析以减小相关性的影响。考虑到因子数量的选择带有一定主观,选用主成分分析。
\\\indent 通过主成分分离出独立程度最大的主成分。5个主成分的贡献率如下。
\begin{align}
	\begin{bmatrix}
		\lambda_1\\
		\lambda_2\\
		\lambda_3\\
		\lambda_4\\
		\lambda_5
	\end{bmatrix}
	=
	\begin{bmatrix}
		4.5349 \\
		0.4149 \\
		0.0497 \\
		0.0006 \\
		0
	\end{bmatrix}
\end{align}
\par\indent 可以发现只有前2个主成分贡献率较大,且远大于其他主成分之和。故将其他主成分视为随机干扰,忽略其影响。
\begin{align}
	\hat{y}=0.5328z_1+1.2430z_2
\end{align}
\par\indent 主成分\(z_1,z_2\)与\(\tilde{x}_1..\tilde{x}_5\)的关系如下。
\begin{align}
	z_1&=0.5641\tilde{x}_1-0.0749\tilde{x}_2-0.2598\tilde{x}_3-0.0379\tilde{x}_4-0.0021\tilde{x}_5\\
	z_2&=3.7058\tilde{x}_1-0.5204\tilde{x}_2+1.1096\tilde{x}_3+0.0113\tilde{x}_4+0.0007\tilde{x}_5
\end{align}
\par\indent 只保留前2个主成分后\(y\)与\(\tilde{x}_1..\tilde{x}_5\)的关系如下。
\begin{align}
	\hat{y}=4.9069\tilde{x}_1-0.6868\tilde{x}_2+1.2408\tilde{x}_3-0.0061\tilde{x}_4-0.0002\tilde{x}_5
\end{align}
\begin{enumerate}
	\item 第一个主成分跟研发显著正相关,而跟其他环节轻微负相关,原因可能是其他环节的投入增加会限制研发的投入,不妨称为创新主成分;
	\item 第二个主成分跟研发和销售有一定关系,而跟其他环节有较少的相关性。考虑到为了迎合顾客喜好,研发设计特殊的机型,以及对不同的顾客实行不同的销售手段和特殊的宣传策略,所以该主成分可能与之相关,不妨称之为顾客喜好主成分。
\end{enumerate}
\par\indent 根据求解模型得到的结果,企业在对5个环节投入总和有限的情况下,应将全部财力、人力投入到创新上,即按照主成分对应每一个环节的比例分配投入,但投入不能为负值,所以因将投入集中到研发上。
\\\indent 但这只是根据统计数据得到的分析,现实中没有其他环节的投入无法支撑一个企业的存在。求解的结果只能反映研发与手机这一行业的利润密切相关,从侧面印证了手机属于高新技术产业。
\\\indent 综上所述,应加大在研发上的投入,其次应考虑对目标顾客进行针对性的营销策略。
\subsection{模型的检验与分析}
\subsubsection{相关性分析}
为了分析变量\(x_i,y\)的相关性的大小,需要求出拟合优度\(R^2\)。
\begin{align}
	R^2=\frac{\sum(\hat{y}_i-\bar{y})^2}{\sum(y_i-\bar{y})^2}
\end{align}
\par\indent\(R^2\)为0.832,说明\(x_i,y\)显著相关。
\subsubsection{F检验}
拟合优度只能说明拟合程度的好坏,还需要检验整个模型是否显著。
\\\indent 提出假设\(H_0\):回归系数全为0;\(H_1\):回归系数不全为0。
\begin{align}
	F=\frac{\sum(\hat{y}_i-\bar{y})^2}{1}\Big/\frac{\sum(\hat{y}_i-y_i)^2}{n-2}
\end{align}
\par\indent 给定显著性水平\(\alpha=0.05\)。\(F=34.61\geqslant F_\alpha(1,n-2)\)。拒绝\(H_0\)。即回归方程是显著的。
\subsubsection{t检验}
进一步对回归系数进行显著性检验,如果不显著,则认为模型不成立。
\\\indent 提出假设\(H_0\):\(x_1\)的回归系数\(a_1\)的为0;\(H_1\):\(x_1\)的回归系数不为0。
\begin{align}
	t_1&=\lVert\frac{\hat{a}_1}{S_1}\rVert\\
	S_1&=\frac{\sigma^2}{\sqrt{\sum(x-\bar{x})^2}}
\end{align}
\par\indent 其中,\(\sigma^2\)是估计的方差。给定显著性水平\(\alpha=0.05\)。\(t_1=53.76\geqslant t_{\alpha/2}(n-2)\)。拒绝\(H_0\)。即回归系数是显著的。
\\\indent 对于其它项可同理分析。
%-------------------------------------------------------------------------------------------------------
\ifnum\strcmp{\jobname}{PCA}=0
	\end{document}
\fi