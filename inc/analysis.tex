\ifnum\strcmp{\jobname}{analysis}=0
	\input{../set/preclass}
	\documentclass{article}
	\input{../set/default}
	\input{../set/user}
	\makeatletter
%-------------------------------------------------------------------------------------------------------
\usepackage{enumitem}
\setlist[enumerate, 1]
{	fullwidth,
	label = \arabic*., 
	font = \textup, 
	itemindent=2em
}
\setlist[enumerate, 2]
{	fullwidth,
	label = \arabic*., 
	font = \textup, 
	itemindent=2em
}
%-------------------------------------------------------------------------------------------------------
\pagestyle{fancy}
\renewcommand{\headrulewidth}{0pt}
\lhead{}
\chead{}
\rhead{}
\lfoot{\small{第\chinese{section}节}}
\cfoot{\small{第\thepage 页~共\pageref{LastPage}页}}
\rfoot{}
%-------------------------------------------------------------------------------------------------------
\title{\textbf{基于统计学方法的企业经营优化}}
\author{}
\date{}
%-------------------------------------------------------------------------------------------------------
\makeatother

	\begin{document}
\fi
%-------------------------------------------------------------------------------------------------------
\clearpage
\section{问题分析}
\subsection{假设}
\begin{enumerate}
	\item 研发设计的投入对于利润的贡献具有滞后性,假设其他环节的投入如生产制造、售后服务等都具有及时性;
	\item 预定目标人群包括学生、怀旧、注重设计感的群体,主体年龄阶段假设为20岁以下以及60岁以上;
	\item 假设公司的投资决策具有一定的稳定性,在一段时间内不变;
	\item 由于安全事故发生概率极小且对利润影响极大因此假设安全环节投入足够且一段时间内无安全事故的发生。
\end{enumerate}
\subsection{标记}
\begin{table}[htbp]
	\centering
	\caption{变量的定义}
	\begin{tabu}spread0pt{X[c]X[l]}
		\toprule
		变量&	定义\\
		\midrule
		\(y\)&利润\\
		\(x_i\)&第\(i\)个环节的综合投入\\
		\(x_{i1}\)&第\(i\)个环节的财力投入\\
		\(x_{i2}\)&第\(i\)个环节的人员投入\\
		\(t_i\)&第\(i\)个年龄段的顾客比例\\
		\bottomrule
	\end{tabu}
\end{table}
\subsection{分析}
\subsubsection{题一}
针对问题一,我们将新产品的推出分为研发设计、生产制造、营销、管理以及售后服务等环节,每个环节包含投入占比和人员占比两个变量,由熵权法确定各个环节理论上最优组合,然后通过主成分分析法确定各环节的投入产出比,得到最终的利润模型。考虑到研发设计的投入与最终贡献具有滞后性,因此采用时间序列回归分析确定研发设计的投入产出比。针对各个环节的投入产出比,合理调整投入分配,可以得到最大利润。
\subsubsection{题二}
针对问题二,首先根据Nokia购买人群的分布图得出实际购买人群年龄阶段,与预设购买人群进行对比。针对实际购买人群与预设购买人群的差异,调整各个环节的投入产出比,再根据各个环节的投入产出比调整环节投入,优化模型一获得最大利润。
\subsubsection{题三}
针对问题三,主要是利用数学模型评价“香蕉机”的市场前景,合理展示本公司的投入分配,重点强调投入的合理性以及利润的可观,综合预期市场和企业定位,强化社会融资和银行贷款的信心。
%-------------------------------------------------------------------------------------------------------
\ifnum\strcmp{\jobname}{analysis}=0
	\end{document}
\fi
