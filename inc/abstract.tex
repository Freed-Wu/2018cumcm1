\ifnum\strcmp{\jobname}{abstract}=0
	\input{../set/preclass}
	\documentclass{article}
	\input{../set/default}
	\input{../set/user}
	\makeatletter
%-------------------------------------------------------------------------------------------------------
\usepackage{enumitem}
\setlist[enumerate, 1]
{	fullwidth,
	label = \arabic*., 
	font = \textup, 
	itemindent=2em
}
\setlist[enumerate, 2]
{	fullwidth,
	label = \arabic*., 
	font = \textup, 
	itemindent=2em
}
%-------------------------------------------------------------------------------------------------------
\pagestyle{fancy}
\renewcommand{\headrulewidth}{0pt}
\lhead{}
\chead{}
\rhead{}
\lfoot{\small{第\chinese{section}节}}
\cfoot{\small{第\thepage 页~共\pageref{LastPage}页}}
\rfoot{}
%-------------------------------------------------------------------------------------------------------
\title{\textbf{基于统计学方法的企业经营优化}}
\author{}
\date{}
%-------------------------------------------------------------------------------------------------------
\makeatother

	\begin{document}
\fi
%-------------------------------------------------------------------------------------------------------
\setcounter{page}{1}
\maketitle
\thispagestyle{empty}
\begin{abstract}
	企业资金分配与管理是企业经营的支柱性决策,针对不同产品、不同市场定位,企业可以合理调整资金投入计划包括产品研发、生产制造、营销、售后、管理等环节,以期获得最大利润。
	\\\indent 针对问题一,将新产品推向市场分为研发设计、生产制造、营销、管理以及售后服务等五个环节,每个环节包括投入占比和人员占比两个变量,各环节的投入量和投入产出比共同影响了利润。首先确定理想投入分配比例和贡献率,选用层次分析法、熵权法、神经网络三种方法进行比较,由于层次分析法过于主观不符合客观要求、由于各个公司同一标志相差不明显导致熵权法所得权值与实际不相符合,因此最终选择神经网络求得各个环节理想投入分配量;由于五个环节之间并不是相互独立的,因此建立主成分分析模型减小各环节相关性的影响,将五个环节综合成创新主成分和顾客喜好主成分两个部分。根据模型结果可知应加大研发上的投入,其次是针对预设购买人群进行针对性的营销策略。模型结果也符合手机生产厂商高科技企业的定位。
	\\\indent 针对问题二,首先对比实际购买人群分布图与预设购买人群分布图,发现实际购买人群以中年、老年群体为主,学生群体占比明显小于预设购买量。考虑到实际市场反馈情况以及不同年龄人群消费力水平不同对利润的影响,因此建立多层线性回归模型,第一层为购买人群年龄结构与各环节投入的关系,第二层为购买人群年龄结构与利润的关系,
根据两个线性回归模型可以调整各环节投入使消费水平最高的群体为预期购买人群,优化了(1)中的模型获得最大利润。
	\\\indent 针对问题三,用数学模型分析了香蕉机的预期市场前景和购买人群分布,介绍针对香蕉机预期市场情况本公司做出的投入分配计划,重点强调了增加研发设计的投入以及根据市场反馈制定的营销策略的投入,展示各环节利润贡献率以及调整方案,展示企业未来的盈利预期,增强社会融资和银行贷款信心。
	\\\textbf{关键词:}企业经营;投入分配;购买人群分布;利润最大化。
\end{abstract}
\clearpage
%\begin{center}
%	\normalsize\textbf{Summary}
%\end{center}
%\par\small
%\begin{spacing}{0.5}
%	\\\textbf{Keywords: }
%\end{spacing}
%\normalsize
%\clearpage
%-------------------------------------------------------------------------------------------------------
\ifnum\strcmp{\jobname}{abstract}=0
	\end{document}
\fi
