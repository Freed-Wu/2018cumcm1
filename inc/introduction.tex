\ifnum\strcmp{\jobname}{introduction}=0
	\input{../set/preclass}
	\documentclass{article}
	\input{../set/default}
	\input{../set/user}
	\makeatletter
%-------------------------------------------------------------------------------------------------------
\usepackage{enumitem}
\setlist[enumerate, 1]
{	fullwidth,
	label = \arabic*., 
	font = \textup, 
	itemindent=2em
}
\setlist[enumerate, 2]
{	fullwidth,
	label = \arabic*., 
	font = \textup, 
	itemindent=2em
}
%-------------------------------------------------------------------------------------------------------
\pagestyle{fancy}
\renewcommand{\headrulewidth}{0pt}
\lhead{}
\chead{}
\rhead{}
\lfoot{\small{第\chinese{section}节}}
\cfoot{\small{第\thepage 页~共\pageref{LastPage}页}}
\rfoot{}
%-------------------------------------------------------------------------------------------------------
\title{\textbf{基于统计学方法的企业经营优化}}
\author{}
\date{}
%-------------------------------------------------------------------------------------------------------
\makeatother

	\begin{document}
\fi
%-------------------------------------------------------------------------------------------------------
\clearpage
\setcounter{section}{-1}
\section{引言}
\subsection{问题背景}
智能手机已经成为多数人生活中必不可少之物,曾经是手机行业翘楚的诺基亚一度因为对市场的误判消失在大众视野中,但是,诺基亚在悄悄回归。在MWC 2018大展上,诺基亚正式宣布了8110经典机型的复刻计划,机身采用弧形波浪的时尚设计。在营销中,主推黄色,被昵称为‘香蕉机',较受瞩目和追捧。
\\\indent 硬件配置上,诺基亚8110是一部搭载2.4英寸QVGA彩色屏幕和200万像素后置摄像头的手机,支持双卡和LTE连接,功能机OS内置应用商店和新版贪食蛇游戏,最长待机时间可达25天。新的4G复刻版依然是下滑盖,机身更薄,携带方便。名为复古,实际上做了重新设计,屏幕更大、调整键盘布局,支持4G LTE,内置地图、流行社交软件等应用。该款手机目前开始在亚洲上市销售,售价远低于流行的智能手机。
\\\indent 任何一种新产品投放市场,各种变化与不确定性都将检验其投放市场前的预判,需要根据实际市场反馈和企业未来发展进行调整。
\subsection{问题信息}
\begin{enumerate}
	\item 诺基亚推出的“香蕉机”增强了待机时长、双卡等硬件设施和商务用途,削弱了触屏、游戏等娱乐化功能,主要面向商务用途、复古怀旧等人群。主要优点包括复古设计、待机时间长、价格低廉;
	\item 新产品的利润影响环节包括设计研发、生产制造、营销、管理以及售后服务等环节,每个环节投入产出比不同,合理优化每个环节的投入能够获得最大利润;
	\item 根据实际市场反馈与企业未来发展,可以对新产品各个环节投入进行调整更加适应市场需要;
\end{enumerate}
\subsection{问题重述}
\begin{enumerate}
	\item 试从企业的产品设计、生产制造、营销、测试、维护、研发、安全等环节,建立数学模型,说明企业如何管理来实现利润最大化;
	\item 针对诺基亚推出的Nokia 8110香蕉机,适当减弱了娱乐功能,强调实用功能,其预设购买对象为怀旧、注重设计感、学生等人群。经过一段时间销售后,调查分析实际购买人群与预设人群是否存在差异,优化问题(1)中的模型,使得企业的利润获取最大;
	\item 作为诺基亚公司董事会的一名高层管理者,请写出一份(不超过两页)新推Nokia 8110香蕉机的全面评估报告,强化向社会融资和银行贷款的信心;
\end{enumerate}
%-------------------------------------------------------------------------------------------------------
\ifnum\strcmp{\jobname}{introduction}=0
	\end{document}
\fi
