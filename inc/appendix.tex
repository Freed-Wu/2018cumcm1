\ifnum\strcmp{\jobname}{appendix}=0
	\input{../set/preclass}
	\documentclass{article}
	\input{../set/default}
	\input{../set/user}
	\makeatletter
%-------------------------------------------------------------------------------------------------------
\usepackage{enumitem}
\setlist[enumerate, 1]
{	fullwidth,
	label = \arabic*., 
	font = \textup, 
	itemindent=2em
}
\setlist[enumerate, 2]
{	fullwidth,
	label = \arabic*., 
	font = \textup, 
	itemindent=2em
}
%-------------------------------------------------------------------------------------------------------
\pagestyle{fancy}
\renewcommand{\headrulewidth}{0pt}
\lhead{}
\chead{}
\rhead{}
\lfoot{\small{第\chinese{section}节}}
\cfoot{\small{第\thepage 页~共\pageref{LastPage}页}}
\rfoot{}
%-------------------------------------------------------------------------------------------------------
\title{\textbf{基于统计学方法的企业经营优化}}
\author{}
\date{}
%-------------------------------------------------------------------------------------------------------
\makeatother

	\begin{document}
\fi
%-------------------------------------------------------------------------------------------------------
\clearpage
\addcontentsline{toc}{section}{附录}
\begin{appendix}
	\section{Nokia 8110香蕉机的全面评估报告}
	Nokia 8110作为我们公司最新推出的产品,具有弧形波浪机身的时尚设计,配以明亮的黄色,更添动感与时尚气息。在硬件配置上,此款机型搭载2.4英寸QVGA彩色屏幕和200万像素后置摄像头的手机,支持双卡和LTE连接,功能机OS内置应用商店和新版贪食蛇游戏,最长待机时间可达25天。新的4G复刻版依然是下滑盖,机身更薄,携带方便。名为复古,实际上做了重新设计,屏幕更大、调整键盘布局,支持4G LTE,内置地图、流行社交软件等应用。最重要的是售价远远低于其他智能手机,非常适合商务人群以及学生群体。
	\\\indent 为了更有针对性的推广Nokia 8110,我们对市场做了严格而准确的分析,发现Nokia 8110具有十分广阔的市场前景与利润空间。
	\\\indent 首先我们评估了一般情况下公司对于产品研发设计、生产制造、营销、售后服务以及行政管理等五个环节投入分配情况与利润的关系,通过神经网络算法和主成分分析法我们得出了两个主要方案,一是加大研发上的投入,增强产品自身特点与辨识度,增强用户黏性;其次是针对预期购买人群的进行合理营销,吸引潜在用户。而这两种方案也恰恰是我们公司一直以来全力贯彻的方案,注重产品研发设计,注重客户需求与体验性,因此从一般意义上讲新款Nokia 8110具有什么广阔的市场前景与利润空间。
	\\\indent 其次,由于不同年龄层次的顾客购买能力以及对产品和服务的需求不同,因此我们着重研究了研发设计、生产制造、营销、售后和管理等五个环节的投入分配与实际购买人群年龄层次的关系以及不同的实际购买人群年龄层次分布对于利润的影响。经过严格计算分析,我们发现19-28岁年龄阶段购买力高于其他年龄阶段人群。由于此款机型的自身特点同样适用于青年人群体中部分怀旧复古的人群,因此我们对五个环节的投入做出了调整:第一是加大研发设计投入,针对青年群体的喜好进一步加强机身、颜色的时尚性设计;第二是加大营销投入,重点在青年人聚集的平台上加强产品推广;第三是加大生产制造方面的投入保证现货,更适应于年轻人不愿意等待的普遍心理。
	\\\indent 针对Nokia 8110独特设计和市场定位,我们合理调整了公司运营与投入决策,更加符合市场需求,也因此创造了更广阔的市场前景和利润空间。同时,由于模型良好的扩展性,对于公司未来的产品策略也同样适用,因此公司的发展前景也同样非常广阔。
	\clearpage
	\section{代码}
	\subsection{主成分分析}
	\lstinputlisting
	[	language=Matlab,
		caption=熵权法.m
	]{src/entropymethod.m}
	\lstinputlisting
	[	language=Matlab,
		caption=主成分分析.m
	]{src/PCA.m}
	\subsection{多层线性回归}
	\lstinputlisting
	[	language=Matlab,
		caption=多层线性回归.m
	]{src/HLM.m}
\end{appendix}
%-------------------------------------------------------------------------------------------------------
\ifnum\strcmp{\jobname}{appendix}=0
	\end{document}
\fi
