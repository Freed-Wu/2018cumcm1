\makeatletter
%-------------------------------------------------------------------------------------------------------
\xdefinecolor{dkgreen}{rgb}{0, 0.6, 0}
\xdefinecolor{gray}{rgb}{0.5, 0.5, 0.5}
\xdefinecolor{mauve}{rgb}{0.58, 0, 0.82}
\xdefinecolor{DeepSkyBlue}{RGB}{0, 104, 139}
\xdefinecolor{dkpurple}{rgb}{0.455, 0.204, 0.506}
\newif\ifTJ@green
\newif\ifTJ@orange
\newif\ifTJ@violet
\TJ@greentrue
\TJ@orangefalse
\TJ@violetfalse
\ifTJ@green
	\xdefinecolor{cvgreen}{HTML}{92D14F}
	\xdefinecolor{cvgray}{HTML}{D8E4BE}
	\xdefinecolor{cvtext}{HTML}{92909B}
\else
\fi
\ifTJ@orange
	\xdefinecolor{cvgreen}{RGB}{230, 140, 20}
	\xdefinecolor{cvgray}{RGB}{248, 216, 174}
	\xdefinecolor{cvtext}{RGB}{100, 100, 100}
\else
\fi
\ifTJ@violet
	\xdefinecolor{cvgreen}{RGB}{142, 10, 178}
	\xdefinecolor{cvgray}{RGB}{209, 181, 217}
	\xdefinecolor{cvtext}{RGB}{100, 100, 100}
\else
\fi
%-------------------------------------------------------------------------------------------------------
\usepackage{ctex}
\xeCJKsetup
{	CJKecglue = {\hskip 0.10em plus 0.05em minus 0.05em}, 
	CheckSingle = true
}
\setCJKmainfont
[	BoldFont = {Noto Serif CJK SC Bold}, 
	ItalicFont = {AR PL UKai HK}
]{Noto Serif CJK SC ExtraLight}
\setCJKsansfont
[	BoldFont = {Noto Sans S Chinese Medium}, 
%	ItalicFont = {Noto Sans S Chinese DemiLight}
]{SimSun}
\setCJKmonofont
[	BoldFont = {Adobe Heiti Std}, 
	ItalicFont = {Adobe Kaiti Std}
]{WenQuanYi Micro Hei Mono Light}
\setCJKfamilyfont{zhkesong}{MF KeSong (Noncommercial)}
\setCJKfamilyfont{zhhjsd}{MF TheGoldenEra (Noncommercial)}
\setCJKfamilyfont{zhlangsong}{MF LangSong (Noncommercial)}
\newcommand{\hjsd}{\CJKfamily{zhhjsd}}
\newcommand{\KeSong}{\CJKfamily{zhkesong}}
\newcommand{\langsong}{\CJKfamily{zhlangsong}}
\newfontfamily\KeSongf[Mapping = tex-text]{MF KeSong (Noncommercial)}
\newfontfamily\hjsdf[Mapping = tex-text]{MF TheGoldenEra (Noncommercial)}
\newfontfamily\langsongf[Mapping = tex-text]{MF LangSong (Noncommercial)}
\newfontfamily\SourceCodePro[Mapping = tex-text]{Source Code Pro}
\newfontfamily\monaco{Monaco}
%-------------------------------------------------------------------------------------------------------
\usepackage{microtype}
%-------------------------------------------------------------------------------------------------------
\usepackage
[	colorlinks = true, 
	linkcolor = gray, 
	citecolor = gray,
	backref=page
]{hyperref}
%-------------------------------------------------------------------------------------------------------
\renewcommand*\lstlistingname{代码}
\lstset
{	language = Matlab, 			%thelanguageofthecode
	basicstyle =\tiny\monaco, 		%thesizeofthefontsthatareusedforthecode
	numbers = left, 				%wheretoputtheline-numbers
	numberstyle = \tiny\color{gray}\monaco, 	%thestylethatisusedfortheline-numbers
	stepnumber = 1, 			%thestepbetweentwoline-numbers.Ifit's1, eachlinewillbenumbered
	numbersep = 5pt, 			%howfartheline-numbersarefromthecode
	backgroundcolor = \color{white}, %choosethebackgroundcolor.Youmustadd\usepackage{color}
	showspaces = false, 			%showspacesaddingparticularunderscores
	showstringspaces = false, 		%underlinespaceswithinstrings
	showtabs = false, 			%showtabswithinstringsaddingparticularunderscores
	frame = single, 				%addsaframearoundthecode
	rulecolor = \color{black}, 		%ifnotset, theframe-colormaybechangedonline-breakswithinnot-blacktext(e.g.commens(greenhere))
	tabsize = 4, 				%setsdefaulttabsizeto2spaces
	captionpos = b, 				%setsthecaption-positiontobottom
	breaklines = true, 			%setsautomaticlinebreaking
	breakatwhitespace = false, 	%setsifautomaticbreaksshouldonlyhappenatwhitespace
	title = \lstname, 				%showthefilenameoffilesincludedwith\lstinputlisting; alsotrycaptioninsteadoftitle
	keywordstyle = \color{blue}, 	%keywordstyle
	commentstyle = \color{dkgreen}, %commentstyle
	stringstyle = \color{mauve}, 	%stringliteralstyle
	escapeinside = ``, 			%ifyouwanttoaddLaTeXwithinyourcode
	morekeywords = {*, ...},		%ifyouwanttoaddmorekeywordstotheset
	columns=flexible
}
%-------------------------------------------------------------------------------------------------------
\tcbset
{	colframe = yellow! 50! green, 
	colback = white, 
	fonttitle = \bfseries, 
	left = 0mm, 
	right = 0mm, 
	top = 0mm, 
	bottom = 0mm, 
	boxsep = 0mm, 
 	toptitle = 0.5mm, 
 	bottomtitle = 0.5mm, 
	nobeforeafter, center title
}
\newtcbtheorem{strength}{\it 优点}
{	enhanced, breakable, 
	colback = white, 
	colframe = cyan, 
	colbacktitle = cyan, 
	attach boxed title to top left = 
	{	yshift = -2mm, 
		xshift = 5mm
	}, 
	boxed title style = {sharp corners}, 
	fonttitle = \sffamily\bfseries, 
	separator sign = {\it :}
}{}
\newtcbtheorem{weakness}{\it 缺点}
{	enhanced, breakable, 
	colback = white, 
	colframe = gray, 
	colbacktitle = gray, 
	attach boxed title to top left = 
	{	yshift = -2mm, 
		xshift = 5mm
	}, 
	boxed title style = {sharp corners}, 
	fonttitle = \sffamily\bfseries, 
	separator sign = {\it :}
}{}
%-------------------------------------------------------------------------------------------------------
\tikzstyle{level 2 concept} += [sibling angle = 40]
\tikzstyle{startstop} = 
[	rectangle, rounded corners, 
	minimum width = 2cm, 
	minimum height = 1cm, 
	text centered, 
	draw = black, 
	fill = red! 40
]
\tikzstyle{io} = 
[	trapezium, 
	trapezium left angle = 70, 
	trapezium right angle = 110, 
	minimum width = 2cm, 
	minimum height = 1cm, 
	text centered, 
	draw = black, 
	fill = blue! 40
]
\tikzstyle{process} = 
[	rectangle, 
	minimum width = 2cm, 
	minimum height = 1cm, 
	text centered, 
	draw = black, 
	fill = yellow! 50
]
\tikzstyle{decision} = 
[	diamond, 
	aspect = 3, 
	text centered, 
	draw = black, 
	fill = green! 30
]
\tikzstyle{arrow} = 
[	->, 
	> = stealth
]
%-------------------------------------------------------------------------------------------------------
\newcommand{\drawbarcode}[1]{\EANisbn[SC5a, ISBN = #1]}
%-------------------------------------------------------------------------------------------------------
\renewcommand{\thefootnote}{\fnsymbol{footnote}}
%-------------------------------------------------------------------------------------------------------
\usepackage[resetfonts]{cmap}
\usepackage{fancyvrb}
\begin{VerbatimOut}{main.cmap}
	%!PS-Adobe-3.0 Resource-CMap
	%%DocumentNeededResources: ProcSet (CIDInit)
	%%IncludeResource: ProcSet (CIDInit)
	%%BeginResource: CMap (TeX-OT1-0)
	%%Title: (TeX-OT1-0 TeX OT1 0)
	%%Version: 1.000
	%%EndComments
	/CIDInit /ProcSet findresource begin
	12 dict begin
	begincmap
	/CIDSystemInfo
	<< /Registry (TeX)
	/Ordering (OT1)
	/Supplement 0
	>> def
	/CMapName /TeX-OT1-0 def
	/CMapType 2 def
	1 begincodespacerange
	<00> <7F>
	endcodespacerange
	8 beginbfrange
	<00> <01> <0000>
	<09> <0A> <0000>
	<23> <26> <0000>
	<28> <3B> <0000>
	<3F> <5B> <0000>
	<5D> <5E> <0000>
	<61> <7A> <0000>
	<7B> <7C> <0000>
	endbfrange
	40 beginbfchar
	<02> <0000>
	<03> <0000>
	<04> <0000>
	<05> <0000>
	<06> <0000>
	<07> <0000>
	<08> <0000>
	<0B> <0000>
	<0C> <0000>
	<0D> <0000>
	<0E> <0000>
	<0F> <0000>
	<10> <0000>
	<11> <0000>
	<12> <0000>
	<13> <0000>
	<14> <0000>
	<15> <0000>
	<16> <0000>
	<17> <0000>
	<18> <0000>
	<19> <0000>
	<1A> <0000>
	<1B> <0000>
	<1C> <0000>
	<1D> <0000>
	<1E> <0000>
	<1F> <0000>
	<21> <0000>
	<22> <0000>
	<27> <0000>
	<3C> <0000>
	<3D> <0000>
	<3E> <0000>
	<5C> <0000>
	<5F> <0000>
	<60> <0000>
	<7D> <0000>
	<7E> <0000>
	<7F> <0000>
	endbfchar
	endcmap
	CMapName currentdict /CMap defineresource pop
	end
	end
	%%EndResource
	%%EOF
\end{VerbatimOut}
%-------------------------------------------------------------------------------------------------------
\makeatother
